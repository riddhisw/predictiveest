\documentclass[pra, reprint]{revtex4-1}

\usepackage{amssymb,amsmath}
\usepackage{graphicx, import}
% \usepackage{caption}  # not compatible with Revtek
% \usepackage{subcaption}  # not compatible with Revtek
\usepackage{placeins}
\usepackage{hyperref}
\usepackage[nameinlink]{cleveref}

\graphicspath{ {./pdfs/}} %{/home/riddhisw/Documents/2017/Scripts_Git/v0/FIGS_v0/}}

% Clever Referencing -- replace with \ref at the every end
\crefname{equation}{Eq.}{Eqs.}
\crefname{figure}{FIG.}{FIGS.}
\crefname{table}{Table}{Tables} 
\crefname{section}{Section}{Sections}
\crefname{chapter}{Chapter}{Chapters}
\crefname{appendix}{Appendix}{Appendices}

% Define state  and bayes prediction risk
\newcommand{\state}[0]{f} %mathmode only
\newcommand{\normpr}[0]{\text{N.} \langle (\state_n - \hat{\state}_n)^2 \rangle_{f, \mathcal{D}}} %mathmode
 
% Define Dirac and Pauli notation
\newcommand{\ket}[1]{| #1 \rangle} %mathmode only
\newcommand{\bra}[1]{\langle #1 |} %mathmode only 
\newcommand{\op}[1]{\hat{#1}} %mathmode only - operator hat
\newcommand{\p}[1]{\hat{\sigma}_{#1}} %mathmode only - pauli operators
\newcommand{\idn}[0]{\op{\mathcal{I}}} %mathmode only - identity
\newcommand{\pj}[1]{\ket{#1}\bra{#1}} %mathmode only - projection operator
\newcommand{\pjs}[2]{\bra{#1}#2\rangle} %mathmoe only - state projection

% Define commmon linear operators
\newcommand{\dd}[0]{(t)} %mathmode only - time dependence
\newcommand{\dr}[1]{\frac{d #1}{dt}} %mathmode only - time derivative 
\newcommand{\ex}[1]{\mathbb{E}[#1]} %mathmode only - expectation values
\newcommand{\norm}[1]{||#1||} %mathmode only - norm

% Define apriori and aposteriori estimates for $x$ and $P$ in Kalman Filtering
\newcommand{\amx}[1]{\hat{x}_{#1}(-)} %mathmode only - apriori x at index [1]
\newcommand{\apx}[1]{\hat{x}_{#1}(+)} %mathmode only - aposteriori x at index [1]
\newcommand{\amp}[1]{\hat{P}_{#1}(-)} %mathmode only - apriori P at index [1]
\newcommand{\app}[1]{\hat{P}_{#1}(+)} %mathmode only - aposteriori P at index [1]

%Define format for defintions
\newtheorem{defn}{Definition}
\newtheorem{thm}{Theorem} 
\newtheorem{azm}{Assumption} 

% Define paragraph spacing and line spacing
% \setlength{\parindent}{4em}
\setlength{\parskip}{20em} 
% \renewcommand{\baselinestretch}{1.5}

\begin{document}
With appropriate definitions, the Kalman equations below specify all Kalman algorithms in this paper. At each time step, $n$,  we denote estimates of the moments of the  prior and posterior distributions (equivalently, estimates of the true Kalman state) with $(\amx{n}, \amp{n})$ and $(\apx{n}, \app{n})$ respectively. Any standard textbook (e.g. \cite{grewal2001theory}) yields the Kalman update equations as:

\begin{align}
\amx{n} & = \Phi_{n-1} \apx{n-1} \label{eqn:main:KF:dynamic_x}\\ 
Q_{n-1} & = \sigma^2 \Gamma_{n-1}\Gamma_{n-1}^T  \label{eqn:main:KF:Q}\\
\amp{n}&= \Phi_{n-1} \app{n-1} \Phi_{n-1}^T + Q_{n-1} \label{eqn:main:KF:dynamic_P}\\
\gamma_n &= \amp{n} H_n^T(H_n\amp{n}H_n^T + R_n)^{-1} \label{eqn:main:KF:gain}\\
\hat{y}_n(-) & = h(\amx{n}) \label{eqn:main:KF:step_ahead}\\
\apx{n} &= \amx{n} + \gamma_n (y_n - \hat{y}_n(-)) \label{eqn:main:KF:bayesian_x}\\
\app{n} &= \left[1  - \gamma_n H_n \right] \amp{n} \label{eqn:main:KF:bayesian_p}
\end{align}
To reiterate formally, \cref{eqn:main:KF:dynamic_x} and \cref{eqn:main:KF:dynamic_P} bring the best state of knowledge from the previous time step into the current time step, $n$, as a prior distribution. Dynamical evolution is modified by features of process noise, as encoded in \cref{eqn:main:KF:Q}, and propagated in \cref{eqn:main:KF:dynamic_P}. The propagation of the moments of the apriori distribution, as outlined thus far, does not depend on the incoming measurement, $y_n$, but is determined entirely by the apriori (known) dynamical model, in our case $\Phi \equiv \Phi_n \forall n$. The Kalman gain depends on the uncertainty in the true state, $\amp{n}$ and is modified by features of the measurement model, $H_n$,  and measurement noise, $R\equiv R_n \forall n$. Before seeing any measurement data, the filter predicts an observation $\hat{y}_n(-)$ corresponding to the best available knowledge at $n$ in \cref{eqn:main:KF:step_ahead}. This value is compared to the actual noisy measurement $y_n$ received at $n$, and the difference is used update our knowledge of the true state using \cref{eqn:main:KF:bayesian_x}. If measurement data is noisy and unreliable (high $R$), or if we wish to make $n>1$ step predictions (set $\gamma \equiv 0$), then $\gamma$ has a small or a null value, and the algorithm propagates Kalman state estimates according to the dynamical model and virtually ignores data. In particular, only the second terms in both \cref{eqn:main:KF:bayesian_x} and \cref{eqn:main:KF:bayesian_p} represent the Bayesian update of the moments of priori distribution ($(-)$ terms) to the posterior distribution ($(+)$ terms) at $n$. If $\gamma_n \equiv 0$, then the prior and posterior moments at any time step are exactly identical by \cref{eqn:main:KF:bayesian_x,eqn:main:KF:bayesian_p}, and only dynamical evolution occurs using \cref{eqn:main:KF:dynamic_x,eqn:main:KF:Q,eqn:main:KF:dynamic_P}. 
\end{document}